\newcommand{\cdncdn}[2]{\cfrac{\dif^n #1}{\dif #2^n}}
\newcommand{\cdmcdm}[2]{\cfrac{\dif^m #1}{\dif #2^m}}
\section{Основні поняття та принципи керування}\marginpar{\framebox{14.02.2014}}
{\bf Керування} - це вплив на щось, з цілю підтримки чогось: закону, продукту, тощо. \\
Керування будь-яким об’єктом - це вплив на нього, щоб досягнути певних станів або процесів. Керування об’єктом за допомогою технічних засобів називається {\bf автоматичне керування}. \\
На об’єкт входить кілька входів:\\
\begin{itemize}
\item Керувальний вплив, який впливає на об’єкт з цілю зберігання потрібного значення керувальної величини.
%Російський блок
\item Задающее устройство - устройство, которое задает программу управляющего воздействия, то есть формирует задающий сигнал по управлению.
\end{itemize}
Система управления состоит из ряда функциональных звений. \\
%
Принципи керування:\\
\begin{itemize}
\item Принцип керування за зовнішнім подразнику;
\item Принцип керування за відхиленням.
\end{itemize}
\subsection{Принцип керування за зовнішнім подразником}
Виходом регулятора є керувальний вплив, що подається на об’єкт керування. \\
Якщо якимось чином можна виміряти величину $F$, то можна легко задати $u$. \\
Переваги такої схеми: 
\begin{itemize}
\item Швидкість роботи;
\item Можна добитися повної інваріантності від певних подразників;
\item Не виникає проблем стійкості системи.
\end{itemize}
Недоліки:
\begin{itemize}
\item Велика кількість подразників вимагає відповідної кількості "компенсиционных каналов";
\item Зміна параметрів об’єкта, що керується призводять до появи помилок керування;
\item Можна використовувати лише на тих об’єктах, чиї характеристики точно відомі.
\end{itemize}
\subsection{Принцип керування за відхиленням}
Переваги:
\begin{itemize}
\item Від’ємний зворотній звязок призводить до зменшення помилок незалежно від факторов, які її викликали;
\item Дає найбільшу точність.
\end{itemize}
Недоліки:
\begin{itemize}
\item В цих системах виникає проблема стійкості;
\item В системах неможливо добитися абсолютної незалежності до подразників. Добитися частинної незалежності призводить до ускладення системи та погіршення стійкості (з'являються додаткові зворотні зв’язки);
\end{itemize}
Значним недоліком принципу зворотнього зв’язку є інертність системи. %виділити
Тому часто використовують комбінації цих обох принципів.
%другий слайд
\subsection{Комбінований принцип керування}
Комбінований принцип керування є поєднанням обох принципів. Тобто сигнал (інформація тощо) керування на об’єкт формується двома каналами. Перший, що залежить від відхилення величини, що регулються, від завдання. Другий формує керувальний вплив виключно зі збуджувального сигналу.
\subsection{Поняття лінійного динамічної ланки}
Вивчення властивостей реальних об’єктів керування і систем призводить до опису динамічних ланок у вигляді нелінійних диференційних рівнянь, які можна отримати на основі фізичних (економічних тощо) законів. Система, що формально функціонує, працює у режимі малих відхилень. Тому у цих випадках можна лінеаризувати систему та дослідити цю лінійну систему. А потім врахувати ті особливості, що призводить нелінійність.
%третій слайд
%четвертий слайд
Спектр характеризує співвідношення амплітуд і фаз нескінченно малих синусоїдальних компонент. \\
Часова функція має перетворення Фур’є тільки тоді, коли функція однозначна, містить скінченне число максимумів, мінімумів та розривів і є абсолютно інтегрованою. 
%сьомий слайд
Перетворення Лапласа зображає функцію при $t>0$. Поведінка початкової функції при $t<0$ немає ніякого впливу на функцію Лапласа.\\
В подальшому функція Хевісайда буде використовувати як вхідний вплив. Подібній вплив використовують для вмикання системи та перехід від одного стану до іншого. Змінну $s$ розглядають як оператор диференціювання. Аналогічно вводиться оператор інтегрування.
\subsection{Передавальна функція}
{\bf Передавальною функцією} в операторній формі називається відношення оператору впливу до власного оператору:
\begin{eqnarray}
a_n\cdncdn yt +\ldots+a_1\cdcd yt +a_0y = b_m\cdmcdm ut + \ldots + b_1 \cdcd ut + b_0 u \\
w(p)=\cfrac{B(p)}{A(p)}\\
w(p) = \cfrac{b_mp^m + \ldots + b_1p+b_0}{a_np^n +\ldots +a_1p+a_0}
\end{eqnarray}
Передаточна функція є оператором і її неможливо розглядати як звичайний дріб. Використаємо на цьому рівнянні %вставити посилання
перетворення Лапласа, вважаючи початкові умови нульовими. \\
\begin{eqnarray}
w(s) = \cfrac{b_ms^m + \ldots + b_1s+b_0}{a_ns^n +\ldots +a_1s+a_0}
\end{eqnarray}
Також важлива умова $n\geq m$.\\
Передавальна функція системи (або ланки) в зображенні Лапласа називає систему, що має найменший порядок відношення зображення його вихідної та вхідної змінної при нульових початкових умовах.
\begin{equation}
w(s)=\cfrac{Y(s)}{U(s)}
\end{equation}